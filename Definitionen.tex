\documentclass{article}
\usepackage[left=3cm,right=3cm]{geometry}
\usepackage{lmodern}
\usepackage[ngerman]{babel}
\usepackage[T1]{fontenc}
\usepackage[ansinew]{inputenc}
\input{./Packages}
\usepackage{tikz}
\usepackage{tikz-cd}
\usetikzlibrary{3d}
\usetikzlibrary{calc}
\usetikzlibrary{arrows}
\usetikzlibrary{babel}

\usepackage{amsmath}
\usepackage{amssymb}
\usepackage{nicefrac}


%Provides a list of shortcuts
\newcommand{\Iso}{Isomorphismus}
\newcommand{\Ison}{Isomorphismen}
\newcommand{\Mor}{Morphismus}
\newcommand{\Morn}{Morphismen}

%Provides a list of shortcuts
\providecommand{\Iso}{Isomorphismus\xspace}
\providecommand{\Ison}{Isomorphismen\xspace}
\providecommand{\Mor}{Morphismus\xspace}
\providecommand{\Morn}{Morphismen\xspace}


\begin{document}
	
	%Definition 2.2.2
	Eine \emph{Kategorie} \CatC besteht aus den folgenden Daten:
	\begin{itemize}
		\item einer Klasse $ \Ob \left( \CatC \right)$, deren Elemente wir \emph{Objekte} nennen,
		\item zu je zwei Objekten 
		\begin{math}
			A,B \in \Ob \left( \CatC  \right) 
		\end{math}
		einer Menge 
		\begin{math}
			\Hom_\CatC \left( A,B \right) 
		\end{math}
		, deren Elemente wir mit 
		\begin{math}
			f : A \to B 
		\end{math}
		notieren und \emph{Morphismen} von $ A $ nach $ B $ nennen,
		\item zu je drei Objekten 
		\begin{math}
	 A,B,C \in \Ob \left( \CatC  \right) 
		\end{math}	
		einer \Abb 
		\begin{displaymath}
			\Hom_\CatC \left( A,B \right) \times \Hom_\CatC \left( B,C \right) \to \Hom_\CatC \left( A,C \right) ,
		\end{displaymath}
		die wir mit 
		\begin{math}
			\left( f,g \right) \mapsto g \circ f
		\end{math}
		notieren und \emph{Komposition von Morphismen} nennen, 
\right) \right) 		\item zu jedem Objekt 
		\begin{math}
			A \in \Ob \left( \CatC \right)
		\end{math} 	
		einen ausgezeichneten Morphismus 
		\begin{displaymath}
			\Id_A \in \Hom_\CatC ( A,A) ,
		\end{displaymath}
		welchen wir die \emph{Identit\"at} nennen.
	\end{itemize}
	Diese Daten m\"ussen den folgenden Regeln gen\"ugen:
	\begin{itemize}
		\item Die Komposition von Morphismen ist \emph{assoziativ}: F\"ur drei Morphismen der Form
		\begin{math}
			f: A \to B , g: B \to C, h:C \to D 
		\end{math}
		in \CatC gilt 
		\begin{displaymath}
			h \circ \left( g \circ f \right) = \left( h \circ g \right) \circ f
		\end{displaymath}
		als Morphismen
		\begin{math}
			A \to D.
		\end{math}
		\item Die Identit\"aten sind \emph{beidseitig neutral} \bzgl der Komposition: F\"ur jeden \Mor 
		\begin{math}
			f: A \to B
		\end{math}
		in \CatC gilt
		\begin{displaymath}
			f \circ \Id_A = f = \Id_B \circ f
		\end{displaymath}
	\end{itemize}
	
	%Definition kleine Kategporie
	Eine \emph{kleine Kategorie} ist eine Kategorie, deren Klasse der Objekte eine Menge ist.
	
	% Defininition Fundamentalgruppoid
	
	
	%Definition Isomorphismen
	Sei \KatC eine Kategorie. Ein Morphismus \( f: A \to B  \) heisst  \emph{Isomorphismus}, wenn es einen Morphismus \( g: B \to A \) , sodass \( f \circ g  = \id_B \) und \( g \circ f = \id_B \). 
	Dieser Moprhismus \( g \) ist eindeutig und wird mit \( f^{-1} :B \to A \) bezeichnet.
	\(f^{-1} \) wird auch der zu \( f \)  inverse Morphismus genannt.\\
	Falls es einen Isomorphismus \(A \to B \) gibt sagen wir \( A \) ist isomorph zu \( B \) und schreiben \( A \isom B \).
	
	
				
	 
\end{document}
