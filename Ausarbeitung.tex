!TeX spellcheck = de_DE
!TeX encoding = utf8


\documentclass{article}
\usepackage{standalone}
\usepackage[left=3cm,right=3cm]{geometry}
\usepackage{lmodern}
\usepackage[T1]{fontenc}
\usepackage{standalone}

\usepackage{tikz}
\usepackage{tikz-cd}
\usetikzlibrary{3d}
\usetikzlibrary{calc}
\usetikzlibrary{arrows}
\usetikzlibrary{babel}

\usepackage{amsmath}
\usepackage{amssymb}
\usepackage{nicefrac}


%Provides a list of shortcuts
\usepackage{xspace}
\usepackage{amsmath}
\usepackage{amssymb}
\usepackage{nicefrac}


\providecommand{\catname}[1]{\ensuremath{\mathbf{#1}}\xspace}
\providecommand{\Par}{\catname{Par}}


%Shorthands
\providecommand{\Iso}{Isomorphismus\xspace}
\providecommand{\Ison}{Isomorphismen\xspace}
\providecommand{\Mor}{Morphismus\xspace}
\providecommand{\Morn}{Morphismen\xspace}
\providecommand{\Abb}{Abbildung\xspace}
\providecommand{\Abbn}{Abbildungen\xspace}
%Symbols
\providecommand{\Hom}{\ensuremath{\mathbold{Hom}}}
\providecommand{\Id}{\ensuremath{\mathbb{id}}}

\providecommand{\Ob}{\ensuremath{\mathbold{Ob}}}
%FRAKS
\providecommand{\FA}{\ensuremath{\mathfrak{A}}}
\providecommand{\FB}{\ensuremath{\mathfrak{B}}}
\providecommand{\FC}{\ensuremath{\mathfrak{C}}}
\providecommand{\FD}{\ensuremath{\mathfrak{D}}}
\providecommand{\FE}{\ensuremath{\mathfrak{E}}}
\providecommand{\FF}{\ensuremath{\mathfrak{F}}}
\providecommand{\FG}{\ensuremath{\mathfrak{G}}}
\providecommand{\FH}{\ensuremath{\mathfrak{H}}}
\providecommand{\FI}{\ensuremath{\mathfrak{I}}}
\providecommand{\FJ}{\ensuremath{\mathfrak{J}}}
\providecommand{\FK}{\ensuremath{\mathfrak{K}}}
\providecommand{\FL}{\ensuremath{\mathfrak{L}}}
\providecommand{\FM}{\ensuremath{\mathfrak{M}}}
\providecommand{\FN}{\ensuremath{\mathfrak{N}}}
\providecommand{\FO}{\ensuremath{\mathfrak{O}}}
\providecommand{\FP}{\ensuremath{\mathfrak{P}}}
\providecommand{\FQ}{\ensuremath{\mathfrak{Q}}}
\providecommand{\FR}{\ensuremath{\mathfrak{R}}}
\providecommand{\FS}{\ensuremath{\mathfrak{S}}}
\providecommand{\FT}{\ensuremath{\mathfrak{T}}}
\providecommand{\FU}{\ensuremath{\mathfrak{U}}}
\providecommand{\FV}{\ensuremath{\mathfrak{V}}}
\providecommand{\FW}{\ensuremath{\mathfrak{W}}}
\providecommand{\FX}{\ensuremath{\mathfrak{X}}}
\providecommand{\FY}{\ensuremath{\mathfrak{Y}}}
\providecommand{\FZ}{\ensuremath{\mathfrak{Z}}}
\providecommand{\Fa}{\ensuremath{\mathfrak{a}}}
\providecommand{\Fa}{\ensuremath{\mathfrak{a}}}
\providecommand{\Fb}{\ensuremath{\mathfrak{b}}}
\providecommand{\Fc}{\ensuremath{\mathfrak{c}}}
\providecommand{\Fd}{\ensuremath{\mathfrak{d}}}
\providecommand{\Fe}{\ensuremath{\mathfrak{e}}}
\providecommand{\Ff}{\ensuremath{\mathfrak{f}}}
\providecommand{\Fg}{\ensuremath{\mathfrak{g}}}
\providecommand{\Fh}{\ensuremath{\mathfrak{h}}}
\providecommand{\Fi}{\ensuremath{\mathfrak{i}}}
\providecommand{\Fj}{\ensuremath{\mathfrak{j}}}
\providecommand{\Fk}{\ensuremath{\mathfrak{k}}}
\providecommand{\Fl}{\ensuremath{\mathfrak{l}}}
\providecommand{\Fm}{\ensuremath{\mathfrak{m}}}
\providecommand{\Fn}{\ensuremath{\mathfrak{n}}}
\providecommand{\Fo}{\ensuremath{\mathfrak{o}}}
\providecommand{\Fp}{\ensuremath{\mathfrak{p}}}
\providecommand{\Fq}{\ensuremath{\mathfrak{q}}}
\providecommand{\Fr}{\ensuremath{\mathfrak{r}}}
\providecommand{\Fs}{\ensuremath{\mathfrak{s}}}
\providecommand{\Ft}{\ensuremath{\mathfrak{t}}}
\providecommand{\Fu}{\ensuremath{\mathfrak{u}}}
\providecommand{\Fv}{\ensuremath{\mathfrak{v}}}
\providecommand{\Fw}{\ensuremath{\mathfrak{w}}}
\providecommand{\Fx}{\ensuremath{\mathfrak{x}}}
\providecommand{\Fy}{\ensuremath{\mathfrak{y}}}
\providecommand{\Fz}{\ensuremath{\mathfrak{z}}}

%Provides a list of shortcuts
\usepackage{xspace}
\usepackage{amsmath}
\usepackage{amssymb}
\usepackage{nicefrac}


\providecommand{\catname}[1]{\ensuremath{\mathbf{#1}}\xspace}
\providecommand{\Par}{\catname{Par}}


%Shorthands
\providecommand{\Iso}{Isomorphismus\xspace}
\providecommand{\Ison}{Isomorphismen\xspace}
\providecommand{\Mor}{Morphismus\xspace}
\providecommand{\Morn}{Morphismen\xspace}
\providecommand{\Abb}{Abbildung\xspace}
\providecommand{\Abbn}{Abbildungen\xspace}
%Symbols
\providecommand{\Hom}{\ensuremath{\mathbold{Hom}}}
\providecommand{\Id}{\ensuremath{\mathbb{id}}}

\providecommand{\Ob}{\ensuremath{\mathbold{Ob}}}
%FRAKS
\providecommand{\FA}{\ensuremath{\mathfrak{A}}}
\providecommand{\FB}{\ensuremath{\mathfrak{B}}}
\providecommand{\FC}{\ensuremath{\mathfrak{C}}}
\providecommand{\FD}{\ensuremath{\mathfrak{D}}}
\providecommand{\FE}{\ensuremath{\mathfrak{E}}}
\providecommand{\FF}{\ensuremath{\mathfrak{F}}}
\providecommand{\FG}{\ensuremath{\mathfrak{G}}}
\providecommand{\FH}{\ensuremath{\mathfrak{H}}}
\providecommand{\FI}{\ensuremath{\mathfrak{I}}}
\providecommand{\FJ}{\ensuremath{\mathfrak{J}}}
\providecommand{\FK}{\ensuremath{\mathfrak{K}}}
\providecommand{\FL}{\ensuremath{\mathfrak{L}}}
\providecommand{\FM}{\ensuremath{\mathfrak{M}}}
\providecommand{\FN}{\ensuremath{\mathfrak{N}}}
\providecommand{\FO}{\ensuremath{\mathfrak{O}}}
\providecommand{\FP}{\ensuremath{\mathfrak{P}}}
\providecommand{\FQ}{\ensuremath{\mathfrak{Q}}}
\providecommand{\FR}{\ensuremath{\mathfrak{R}}}
\providecommand{\FS}{\ensuremath{\mathfrak{S}}}
\providecommand{\FT}{\ensuremath{\mathfrak{T}}}
\providecommand{\FU}{\ensuremath{\mathfrak{U}}}
\providecommand{\FV}{\ensuremath{\mathfrak{V}}}
\providecommand{\FW}{\ensuremath{\mathfrak{W}}}
\providecommand{\FX}{\ensuremath{\mathfrak{X}}}
\providecommand{\FY}{\ensuremath{\mathfrak{Y}}}
\providecommand{\FZ}{\ensuremath{\mathfrak{Z}}}
\providecommand{\Fa}{\ensuremath{\mathfrak{a}}}
\providecommand{\Fa}{\ensuremath{\mathfrak{a}}}
\providecommand{\Fb}{\ensuremath{\mathfrak{b}}}
\providecommand{\Fc}{\ensuremath{\mathfrak{c}}}
\providecommand{\Fd}{\ensuremath{\mathfrak{d}}}
\providecommand{\Fe}{\ensuremath{\mathfrak{e}}}
\providecommand{\Ff}{\ensuremath{\mathfrak{f}}}
\providecommand{\Fg}{\ensuremath{\mathfrak{g}}}
\providecommand{\Fh}{\ensuremath{\mathfrak{h}}}
\providecommand{\Fi}{\ensuremath{\mathfrak{i}}}
\providecommand{\Fj}{\ensuremath{\mathfrak{j}}}
\providecommand{\Fk}{\ensuremath{\mathfrak{k}}}
\providecommand{\Fl}{\ensuremath{\mathfrak{l}}}
\providecommand{\Fm}{\ensuremath{\mathfrak{m}}}
\providecommand{\Fn}{\ensuremath{\mathfrak{n}}}
\providecommand{\Fo}{\ensuremath{\mathfrak{o}}}
\providecommand{\Fp}{\ensuremath{\mathfrak{p}}}
\providecommand{\Fq}{\ensuremath{\mathfrak{q}}}
\providecommand{\Fr}{\ensuremath{\mathfrak{r}}}
\providecommand{\Fs}{\ensuremath{\mathfrak{s}}}
\providecommand{\Ft}{\ensuremath{\mathfrak{t}}}
\providecommand{\Fu}{\ensuremath{\mathfrak{u}}}
\providecommand{\Fv}{\ensuremath{\mathfrak{v}}}
\providecommand{\Fw}{\ensuremath{\mathfrak{w}}}
\providecommand{\Fx}{\ensuremath{\mathfrak{x}}}
\providecommand{\Fy}{\ensuremath{\mathfrak{y}}}
\providecommand{\Fz}{\ensuremath{\mathfrak{z}}}

\usepackage[ngerman]{babel} 


\title{Ausarbeitung zum Seminar Kategorientheorie}
\author{Jens Heinrich}
\date{14.04.2016}

\begin{document}
\maketitle
\section{Einleitung}
	Die folgende Ausarbeitung ist als Grundlage und Erg\"anzung zu meinem Vortrag zu sehen. 
	Ich werde einige Beispiele hier genauer ausarbeiten, als ich sie in der Pr\"asentation besprechen werde.
	Was ist Kategorientheorie? 
	Ich sehe die Kategorientheorie ein bisschen als eine Art Hilfsmathematik, so wie andere Wissenschaften die Mathematik als Hilfswissenschaft ansehen;
	die Mathematik an sich scheint in vielen Auspr\"agungen die N\"utzlichkeit f\"ur das reale Leben zu fehlen,
	und so mag sich manchem nicht sofort erschliessen, 
	warum die Kategorientheorie so praktisch ist.
	\\
	Im folgenden Teil werde ich einige Definitionen liefern und sie hoffentlich mit den Beispielen von der N\"utzlichkeit der Defininitionen \"uberzeugen k\"onnen. Der Aufbau folgt dem Aufbau der Quelle \nocite{Bra}.
	
\section{Der Begriff der Kategorie}
	
		\begin{defi}[Kategorie]
		\cite[Definition 2.2.2]{Bra}
		Eine \emph{Kategorie} \CatC besteht aus den folgenden Daten:
		\begin{itemize}
			\item einer Klasse \( \Ob \left( \CatC \right) \), deren Elemente wir \emph{Objekte} nennen,
			\item zu je zwei Objekten 
			\begin{math}
				A,B \in \Ob \left( \CatC  \right) 
			\end{math}
			einer Menge 
			\begin{math}
				\Hom_\CatC \left( A,B \right) 
			\end{math}
			, deren Elemente wir mit 
			\begin{math}
				f : A \to B 
			\end{math}
			notieren und \emph{Morphismen} von $ A $ nach $ B $ nennen,
			\item zu je drei Objekten 
			\begin{math}
		 A,B,C \in \Ob \left( \CatC  \right) 
			\end{math}	
			einer \Abb 
			\begin{displaymath}
				\Hom_\CatC \left( A,B \right) \times \Hom_\CatC \left( B,C \right) \to \Hom_\CatC \left( A,C \right) ,
			\end{displaymath}
			die wir mit 
			\begin{math}
				\left( f,g \right) \mapsto g \circ f
			\end{math}
			notieren und \emph{Komposition von Morphismen} nennen, 
		\item zu jedem Objekt 
			\begin{math}
				A \in \Ob \left( \CatC \right)
			\end{math} 	
			einen ausgezeichneten Morphismus 
			\begin{displaymath}
				\id_A \in \Hom_\CatC ( A,A) ,
			\end{displaymath}
			welchen wir die \emph{Identit\"at} nennen.
		\end{itemize}
		Diese Daten m\"ussen den folgenden Regeln gen\"ugen:
		\begin{itemize}
			\item Die Komposition von Morphismen ist \emph{assoziativ}: F\"ur drei Morphismen der Form
			\begin{math}
				f: A \to B , g: B \to C, h:C \to D 
			\end{math}
			in \CatC gilt 
			\begin{displaymath}
				h \circ \left( g \circ f \right) = \left( h \circ g \right) \circ f
			\end{displaymath}
			als Morphismen
			\begin{math}
				A \to D.
			\end{math}
			\item Die Identit\"aten sind \emph{beidseitig neutral} \bzgl der Komposition: F\"ur jeden \Mor 
			\begin{math}
				f: A \to B
			\end{math}
			in \CatC gilt
			\begin{displaymath}
				f \circ \id_A = f = \id_B \circ f
			\end{displaymath}
		\end{itemize}
		\end{defi}
		Was  diese Definition bedeutet, werde ich nach kurzem Ausholen an einem Beispiel zeigen.
		Hier erstmal noch einige weiter Definitionen:
		
		\begin{defi}[kleine Kategporie] \cite[Definition 2.2.29]{Bra}
		Eine \emph{kleine Kategorie} ist eine Kategorie, deren Klasse der Objekte eine Menge ist.
		\end{defi}
		\begin{defi}[Start und Ziel]  \cite[Bemerkung 2.2.25]{Bra}
		F\"ur einen Morphismus \( f: A \to B \) in einer Kategorie \CatC nennen wir \( A \) das Start- und \( B \) das Ziel-Objekt.
		\end{defi}
		
l		Jetzt folgt das erste Beispiel

		\begin{bsp}
		
	(\cite[Beispiel 2.2.10]{Bra}):
		
		Zu einem K\"orper \( K \) k\"onnen wir die Kategorie der \( K\)-Vectorr\"aume \( \CatVect_K \) betrachten:
		\begin{itemize}
			\item	\( \Ob \left( \CatVect_K \right) = \left\lbrace V | V \text{ist} K-\text{Vektorraum} \right\rbrace  = \left\lbrace K^n | n \in \BN \right\rbrace \)
			\item F\"ur  \( A=K^n , B =K^m , n,m \in \BN \ \text{aus} \ \Ob \left( \CatVect_K) \right) \) existieren $K$-lineare Abbildungen, diese stellen \( \Hom_{\CatVect_K}\left( A,B \right) \) dar
			\item Die Verkn\"upfung entspricht der Matrixmultiplikation der Abbildungsmatrizen 
			\item \( \id_A = E_n \) f\"ur \( K^n , n \in \BN \)
		 \end{itemize}
		 Die f\"ur die Verkn\"upfung von Morphismen geforderten Eigenschaften folgen sofort aus den Eigenschaften der Matrixmultiplikation.
		 \end{bsp}
		 
		 \begin{bsp}
		 Ein weiteres Beispiel ware eine Pr\"aordnung \( X, \leq \) bestehend aus einer Menge \( X \) und einer bin\"aren Relation \( \leq \), welche reflexiv und transitiv ist.
		 
		 \begin{itemize}
			 \item \(\Ob \) = \BN
			 \item \(  \Hom(A,B) = A \leq B \)
			 \item \( A \leq B \wedge B \leq  C \rightarrow A \leq C \)  (da eine Pr\"aordnung transitiv ist) 
			 \item \( \id_A \): \( A \leq A \) (da eine  Pr\"aordnung reflexiv ist)
		 \end{itemize}
			 Das die zus\"atzlichen Gleichungen gelten ist schnell gezeigt:
			 \begin{eqnarray*}
			  \text{Assoziativit\"at:} &  ( x \leq y \wedge x \leq z ) & \wedge z \leq a \\
					 \leftrightarrow & (x \leq z ) & \wedge  z \leq a \\
					 \leftrightarrow & x \leq a & \\
					 \leftrightarrow & 	x \leq y  & \wedge  (y \leq a) \\
					 \leftrightarrow &   x \leq y & \wedge  (x \leq z  \wedge  z \leq a )\\
			 \end{eqnarray}
			 und mit den selben Umformungsschritten gilt dies auch f\"ur die Identit\"aten.
		\end{bsp}
		
		\begin{bsp}
		
		 %Beispiel(Teiler)
		  \cite[Beispiel 2.2.30]{Bra} 
		 \( \Ob(X) = \left\lbrace 1,2,3,6 \right\rbrace \) \\
		 Ein Morphismus \(x   \to y \) existiert, wenn \( x \vert y\) . \\
		 
		  \includestandalone{./files/diagrams/diagram_teiler_6} 
		\end{bsp} 
		 
	 	\begin{defi}[Diskrete und indiskrete Kategorien]
	 	 \cite[Beispiel 2.2.31]{Bra}
	 		Eine diskrete Kategorie ist eine Kategorie, in der die einzigen Morphismen die Identit\"aten sind.
	 		Wenn in einer Kategorie zwischen allen Objekten genau ein Morphismus existiert, sprechen wir von einer indiskreten Kategorie.
	 	\end{defi}	
	 		
	 \begin{bsp}[Datenbanken]
		  \cite[Beispiel 2.2.33]{Bra} \\
		 \includestandalone{./files/diagrams/diagram_database} \\
		 
		 Es wird zus\"atzlich gegeben, dass gilt:
		 \begin{itemize}
			 \item \( \text{arbeitet \ in} \circ \text{Abteilungsleiter}= \id_\text{Abteilung}  \)
			 \item \( \text{Abteilungsleiter} \circ \text{arbeitet \ in} \neq \id_\text{Mitarbeiter}  \)
			 \item \( \text{arbeitet \ in }  \circ \text{Manager} = \text{arbeitet \ in } \)
		 \end{itemize}
	 \end{bsp}
	\newpage
	\section{Isomorphismen}
		
		\begin{defi}[Isomorphismus ]
		 \cite[Definition 2.3.1]{Bra}
		Sei \CatC eine Kategorie. Ein Morphismus \( f: A \to B  \) heisst  \emph{Isomorphismus}, wenn es einen Morphismus \( g: B \to A \) , so dass \( f \circ g  = \id_B \) und \( g \circ f = \id_B \). 
		Dieser Morphismus \( g \) ist eindeutig und wird mit \( f^{-1} :B \to A \) bezeichnet.
		\(f^{-1} \) wird auch der zu \( f \)  inverse Morphismus genannt.\\
		\includestandalone{files/diagrams/diagram_isomorph_objects} \\
		Falls es einen Isomorphismus \( A \to B \) gibt sagen wir \( A \) ist isomorph zu \( B \) und schreiben \( A \equiv B \).
		\end{defi}
		
		
		 \begin{defi}[Gruppoid]
		  \cite[Beispiel 2.2.34]{Bra} \\
		 Ein \emph{Gruppoid} ist eine Kategorie, in der jeder \Mor ein \Iso ist.
		  \end{defi}
		  
		  \begin{lem}[Eigenschaften von Isomorphismen]
		   \cite[Lemma 2.3.9]{Bra}
		  Es sei \CatC eine Kategorie.
		  \begin{enumerate}
			  \item \( \forall A \in \CatC \) ist \( \id_A :A \to A  \) ein Isomorphismus mit \( \id_A^{-1}=\id_A \)
			  \item F\"ur jeden Isomorphismus \( f: A \to B \) in \CatC gilt: Der inverse Morphismus \(f^{-1}: B \to A \) ist ebenfalls ein Isomorphismus mit \( \left( f^{-1} \right)^{-1} \).
			  \item Ist \(g: B \to C \) ein weiterer Isomorphismus, so ist \( g \circ f :A \to C \) ebenfalls ein Isomorphismus mit \( \left( g \circ f \right)^{-1} =f^{-1} \circ g^{-1} \)  
		  \end{enumerate}
		  	\end{lem}
		  	
		 \begin{defi}[Isomorphieklassen]
		  \cite[Definition 2.3.10]{Bra}
		 Da die Isomorphie eine \"Aquivalenzrelation ist k\"onnen, wir von ihren \"Aquivalenzklassen sprechen, welche meist als Isomorphieklassen bezeichnet werden.
		 \end{defi}
		 
		 
		\begin{defi}[Automorphismengruppen]
		\cite[Definition 2.3.13]{Bra}
		 Es sei \CatC eine Kategorie.  \\
		 Ein Isomorphismus \( f:A \to A \) in \CatC heisst auch \emph{Automorphismus} von $A$. 
		 Die Automorphismen von $A$ bilden \bzgl der Komposition eine Gruppe, die \emph{Automorphismusgruppe}.
		\end{defi}
		
		 \begin{defi}[Homotopieklasse]
		  \cite[Aufgabe 2.24]{Bra}
		 Sei X ein topologischer Raum, dann heissen zwei Pfade \( w,w'  \) von \(x \) nach \( y \) homotop, wenn es eine stetige Abbildung \( H : [0,1] \times [0,1 ] \to X \) gibt, sodass \( H(-,0) = w \) , \( H( -,1) =w' \) und \( H(0,t) =x \) und \(H(1,t) = y \forall t \in [0,1] \) gibt.
		 \end{defi}
		 
		 \begin{bsp}[Fundamentalgruppoid]
		 \cite[Aufgabe 2.24 ]{Bra}
		 Sie X ein topologischer Raum
		\begin{itemize}
			\item \(\Ob \left( \Pi \left( X \right) \right) = x \in X \)
			 \item Seien \( x ,y \in X ) \), dann ist \( \Hom(x,y) \) die Homotopieklasse der Pfade von x nach y
			 \item Die Verkn\"upfung der \Hom \ ist das Aneinanderh\"angen von Pfaden.
			 \item \(  \id_x \) ist der konstante Pfad \( w(t)=x \forall t \in [0,1] \)
		\end{itemize}
		Jetzt m\"ussen wir \"uberpr\"ufen, ob die Eigenschaften der Verkn\"upfung gegeben sind.
		Es ist offensichtlich, dass das Vor- oder Nachschalten eines konstanten Weges hier keine Ver\"anderung bringt, somit gilt f\"ur einen beliebigen Weg 
		\(w:x \to y \) mit \( x,y \in X\):
		\( \id_y \circ w = w = w \circ \id_x \) \\
		Die Assoziativit\"at folgt daraus, dass wir die Homtopieklassen betrachten und die Elemente der Homotopieklasse stetig ineinander \"ubergehen.
		Da jeder Weg ein Inverses, gegeben durch \( w^{-1}(t):=w(1-t): [0,1] \to X \) hat, wird diese Kategorie zum Gruppoid.
		Wenn wir jetzt die Automorphismengruppe \( Aut_{\Pi(X)}(x) \) eines Punktes \(x \in X \) aus dem Fundamentalgruppoid betrachten, erhalten wir die Fundamentalgruppe \( \Pi_1(X,x) \) \cite[Beispiel 2.3.14-(8)]{Bra}
		\end{bso}
	
	\section{Kommutative Diagramme}
		\begin{defi}[Kommutatives Diagramm]
		 \cite[Definition 2.4.3]{Bra}
		Sei \( \Gamma =(V,E) \) ein gerichteter Graph. Ein  \emph{Diagramm} $X$ in der Form \( \Gamma \) besteht aus den folgenden Daten:
		\begin{enumerate}
			\item f\"ur jeden Knoten \( v \in V \) ein  Objekt \( X(v) \in \CatC \)
			\item f\"ur jede Kante \( e:v \to w \) in \( E \) ein Morphismus \( X(e):X(v) \to X(w) \)  
		\end{enumerate}
		Das Diagramm $X$ heisst \emph{kommutativ}, wenn f\"ur je zwei Knoten \(v,w \in V \) und je zwei Pfade \( \left(e_n, \dots ,e_1\right) \) und \( \left(f_m, \dots ,f_1\right) \) von $v $ nach $w$ in $\Gamma $ die Gleichung 
		\[
			X(e_n) \circ \dots \circ X(e_1) = X(f_m) \circ \dots \circ X(f_1)
		\]
		von Morphismen \( X(v) \to X(w) \) gilt.
		
		Als Spezialfall hiervon kann man kommutative Dreiecke und Quadrate betrachten:\\
 		\includestandalone{./files/diagrams/diagram_dreieck_easy} 
		\includestandalone{./files/diagrams/diagram_quadrat_easy}	
		\end{defi}
			 
		Ein weiteres sch\"ones Beispiel ist das folgende Diagramm:
		\begin{bsp} \\
		\includestandalone{./files/diagrams/diagram_herz}
		\end{bsp}
		Als nachstes wollen wir uns mit der Diagrammjagd besch\"aftigen \cite[Beispiel 2.4.7]{Bra}.
		Im folgenden Diagramm sind die kleinen Quadrate bereits kommutativ und es ist zu zeigen, dass dann auch schon das grosse Viereck kommutativ ist. \\
		\includestandalone{./files/diagrams/diagram_doppel_quadrat}
		\newpage
		Ein weiteres sch\"ones Beispiel f\"ur eine Diagrammjagd ist  die Kommutativit\"at beim folgenden W\"urfel \cite[Quellcode]{tikzcd} unter der Annahme der Kommutativit\"at der Seitenfl\"achen zu zeigen. \\
		\includestandalone{./files/diagrams/diagram_wuerfel}

	\begin{defi}[Faktorisierung]
	 \cite[Definition 2.4.8]{Bra}
	Eine \emph{Faktorisierung} eines Morphismus \( f:A \to B \) ist ein kommutatives Diagramm \\
		\includestandalone{./files/diagrams/diagram_dreieck_faktor} \\
	. Man sagt $f$ faktorisiert \"uber $g$ bzw. $ C $.
	\end{defi}
	%Bespiel?
\newpage




\section{Initiale und finale Objekte}
\begin{defi}[Initiale und finale Elemente]
 \cite[Definition 2.5.1]{Bra}
	Es sei \CatC eine Kategorie,
	\begin{enumerate}
		\item	 Ein Objekt \( A \in \CatC \) heisst \emph{initial}, 
		falls es f\"ur jedes Element \( B \in \CatC \) genau einen Morphismus \(A \to B \) in \CatC gibt.
		\item	 Ein Objekt \( A \in \CatC \) heisst \emph{final} ( oder auch \emph{terminal}), 
		falls es f\"ur jedes Element \( B \in \CatC \) genau einen Morphismus \(B \to A \) in \CatC gibt.
		\item	Ein Objekt \( A \in \CatC \) heisst \emph{Nullobjekt}, wenn es initial und final ist.
	\end{enumerate}
\end{defi}
	Als Beispiel k\"onnen wir hier wieder die Pr\"aordnung der Teiler von 6 w\"ahlen.
	\begin{bsp}\\
	\includestandalone{./files/diagrams/diagram_teiler_6} \\
	Die \( 1 \in \CatC \) ist initial und  \(6 \in \CatC \) ist terminal, wie man sofort sieht.
	\end{bsp}
	
\begin{lem}[Eindeutigkeit]
 \cite[Lemma 2.5.5]{Bra}
	Es seien \CatC eine Kategorie und 
	\(A , A' \in \CatC \) zwei Objekte.
	\begin{enumerate}
		\item Ist \(A \) initial, so ist \( \End(A)= \left\lbrace \id_A \right\rbrace \) und daher auch \( \Aut(A)=\left\lbrace \id_A \right\rbrace \)
		\item Sind \( A,A' \) initial, so gibt es genau einen Morphismus \( A \to A' \), und dieser ist ein Isomorphismus \( A \equiv A' \) 
		\item Dual dazu sind je zwei finale Objekte einer Kategorie auf eindeutige Weise zueinander isomorph.
		\item Ist \( A \) initial und \( A \equiv A' \), so ist auch \( A' \) initial. Entsprechendes gilt f\"ur finale Objekte.
	\end{enumerate}
	In diesem Sinne sind initiale bzw. finale Objekte also eindeutig.
\end{lem}

\section{Konstruktion mit Kategorien}

\begin{defi}[Unterkategorie] {\cite[Definition 2.6.1]{Bra}} 
Es sei \CatC eine Kategorie und \( K \subset \Ob\left(\CatC\right) \) eine Teilklasse. F\"ur alle \( A,B \in K \) sei eine Teilmenge
\[
	\Hom_\CatD \left( A, B \right) \subset \Hom_\CatC \left( A, B \right)
\]
gegeben. Dabei gelte
\begin{itemize}
	\item \( \id_A \in \Hom_\CatD \left( A, A \right) \forall A \in K \)
	\item \( g \circ f \in  \Hom_\CatD \left( A,C \right) \), falls \( f \in \Hom_\CatD \left( A, B \right) \) und  \( g \in \Hom_\CatD \left( B , C \right) .\)
\end{itemize}
Dann k\"onnen wir eine Kategorie \CatD konstruieren mit \( \Ob \left( \CatD \right) = K \) und Homomorphismus-Mengen \( \Hom_\CatD \left(A,B \right) \) f\"ur \( A,B \in K \). Die Komposition und Identit\"aten werden von \CatC vererbt.
\\
Eine solche Unterkategorie heisst voll, wenn \( \Hom_\CatD \left( A,B \right) = \Hom_\CatC \left(A,B \right) \forall A,B \in K \)
 
\begin{defi}[Duale Kategorie] \cite[Definition 2.6.3]{Bra} 
	Es sei \CatC eine Kategorie.  \\
	Die zu Kategorie \( \CatC^{op} \) wird wie folgt konstruiert:
	\begin{itemize}
		\item \( \Ob \left( \CatC^{op} \right) := \Ob \left( \CatC \right) \)
		\item \( \Hom_{\CatC^{op}} \left( A,B \right) := \Hom_\CatC \left( B,A \right) \forall A,B \in \Ob \left( \CatC \right) \)
		\item  die Identit\"aten ver\"andern sich nicht
		\item \( \circ^{op}:     \Hom_{\CatC^{op}} \left( A,B \right) \times  \Hom_{\CatC^{op}} \left( B,C \right) \to  \Hom_{\CatC^{op}} \left( A,C \right) \)
	\end{itemize}
	Die Verkettung der Morphismen funktioniert wie folgt: \\
	\( \Hom_\CatC \left(B,A \right) \times \Hom_\CatC \left(C,B \right) \equiv \Hom_\CatC \left(C,B \right) \times \Hom_\CatC \left(B,A \right) \overset{in \CatC}{\to} \Hom_\CatC \left(C,A \right) \)
\end{defi}
	Wenn sich die Kategorie als gerichteter Graph ausdr\"ucken l\"asst, dann ist die duale Kategorie der gleiche Graph nur die Richtungen sind umgekehrt.
	
\begin{bsp}
	Hier sehen wir dieses "Umkehren" \ der Pfeile an einem Beispiel: \\
	\includestandalone{./files/diagrams/diagram_teiler_6} 
	\( \leftrightarrow \)
	\includestandalone{./files/diagrams/diagram_teiler_6_dual}
	
\end{bsp}
\begin{defi}[Produkte] \cite[Definition 2.6.6]{Bra} \\
	Sei \( \left( \CatC_i \right)_{ i \in I } \) eine Familie von Kategorien, 
	dann erhalten wir ihr \emph{Produkt} \( \CatD= \prod_{i \in I} \CatC_i }} \) wie folgt:
	\begin{itemize}
		\item \( \Ob \left( \CatD \right) = \prod_{i \in I } \Ob \left(\CatC_i \right) 
		= \left\lbrace \left( X_i \right)_{i \in I } : X_i \in \CatC_i \right) \)
		\item \( \Hom_\CatD \left( X, Y \right) = \prod_{i \in I } \Hom_{\CatC_i} \left( X_i,Y_i \right) \)
	\end{itemize}
	Die Verkettung erfolgt hier komponentenweise.
		
\end{defi}

\begin{defi}[Koprodukte] 
	\cite[Definition 2.6.7]{Bra} \\
	Ist \( \left( X_i \right)_{i \in I} \) eine Familie von Mengen, so ist ihre \emph{disjunkte Vereinigung} oder auch \emph{Koprodukt} durch
	\[
		\coprod_{i \in I } X_i:= \bigcup_{i \in I} X_i \times \left\lbrace i \right\rbrace
	\]
	definiert. \\
	Sei nun \( \left( \CatC_i \right)_{ i \in I } \) eine Familie von Kategorien, 
	dann erhalten wir ihr \emph{Koprodukt} \( \CatD= \coprod_{i \in I} \CatC_i }} \) wie folgt:
	\begin{itemize}
		\item \( \Ob \left( \CatD \right) = \coprod_{i \in I } \Ob \left( \CatC_i \right) \)
		\item \( \Hom_\CatD \left( X, Y \right)  =\left\{\begin{array}{cl} \Hom_{\CatC_i} \left( X_i,Y_i \right) , & \mbox{falls } X_i \in \CatC_i \wedge Y_i \in \CatC_i\\ \varnothing, & \mbox{sonst} \end{array}\right. \)

Es gibt auch noch weitere Arten der Konstruktion, wie \"Uber-A-Kategorie oder Kategorien von Wirkungen, jedoch werde ich auf diese nicht in meinem Vortrag eingehen.

\\
Ich hoffe, dass ich verdeutlichen konnte wie vielseitig und praktisch die Kategorientheorie ist.

%%%% 
\bibliography{sources}
\bibliographystyle{alpha}

\end{document}
