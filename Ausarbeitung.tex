% !TeX spellcheck = de_DE
% !TeX encoding = utf8


\documentclass{article}
\usepackage[left=3cm,right=3cm]{geometry}
\usepackage{lmodern}
\usepackage[ngerman]{babel}
\usepackage[T1]{fontenc}
\usepackage[ansinew]{inputenc}
\usepackage{standalone}

\usepackage{tikz}
\usepackage{tikz-cd}
\usetikzlibrary{3d}
\usetikzlibrary{calc}
\usetikzlibrary{arrows}
\usetikzlibrary{babel}

\usepackage{amsmath}
\usepackage{amssymb}
\usepackage{nicefrac}


%Provides a list of shortcuts
\usepackage{xspace}
\usepackage{amsmath}
\usepackage{amssymb}
\usepackage{nicefrac}


\providecommand{\catname}[1]{\ensuremath{\mathbf{#1}}\xspace}
\providecommand{\Par}{\catname{Par}}


%Shorthands
\providecommand{\Iso}{Isomorphismus\xspace}
\providecommand{\Ison}{Isomorphismen\xspace}
\providecommand{\Mor}{Morphismus\xspace}
\providecommand{\Morn}{Morphismen\xspace}
\providecommand{\Abb}{Abbildung\xspace}
\providecommand{\Abbn}{Abbildungen\xspace}
%Symbols
\providecommand{\Hom}{\ensuremath{\mathbold{Hom}}}
\providecommand{\Id}{\ensuremath{\mathbb{id}}}

\providecommand{\Ob}{\ensuremath{\mathbold{Ob}}}
%FRAKS
\providecommand{\FA}{\ensuremath{\mathfrak{A}}}
\providecommand{\FB}{\ensuremath{\mathfrak{B}}}
\providecommand{\FC}{\ensuremath{\mathfrak{C}}}
\providecommand{\FD}{\ensuremath{\mathfrak{D}}}
\providecommand{\FE}{\ensuremath{\mathfrak{E}}}
\providecommand{\FF}{\ensuremath{\mathfrak{F}}}
\providecommand{\FG}{\ensuremath{\mathfrak{G}}}
\providecommand{\FH}{\ensuremath{\mathfrak{H}}}
\providecommand{\FI}{\ensuremath{\mathfrak{I}}}
\providecommand{\FJ}{\ensuremath{\mathfrak{J}}}
\providecommand{\FK}{\ensuremath{\mathfrak{K}}}
\providecommand{\FL}{\ensuremath{\mathfrak{L}}}
\providecommand{\FM}{\ensuremath{\mathfrak{M}}}
\providecommand{\FN}{\ensuremath{\mathfrak{N}}}
\providecommand{\FO}{\ensuremath{\mathfrak{O}}}
\providecommand{\FP}{\ensuremath{\mathfrak{P}}}
\providecommand{\FQ}{\ensuremath{\mathfrak{Q}}}
\providecommand{\FR}{\ensuremath{\mathfrak{R}}}
\providecommand{\FS}{\ensuremath{\mathfrak{S}}}
\providecommand{\FT}{\ensuremath{\mathfrak{T}}}
\providecommand{\FU}{\ensuremath{\mathfrak{U}}}
\providecommand{\FV}{\ensuremath{\mathfrak{V}}}
\providecommand{\FW}{\ensuremath{\mathfrak{W}}}
\providecommand{\FX}{\ensuremath{\mathfrak{X}}}
\providecommand{\FY}{\ensuremath{\mathfrak{Y}}}
\providecommand{\FZ}{\ensuremath{\mathfrak{Z}}}
\providecommand{\Fa}{\ensuremath{\mathfrak{a}}}
\providecommand{\Fa}{\ensuremath{\mathfrak{a}}}
\providecommand{\Fb}{\ensuremath{\mathfrak{b}}}
\providecommand{\Fc}{\ensuremath{\mathfrak{c}}}
\providecommand{\Fd}{\ensuremath{\mathfrak{d}}}
\providecommand{\Fe}{\ensuremath{\mathfrak{e}}}
\providecommand{\Ff}{\ensuremath{\mathfrak{f}}}
\providecommand{\Fg}{\ensuremath{\mathfrak{g}}}
\providecommand{\Fh}{\ensuremath{\mathfrak{h}}}
\providecommand{\Fi}{\ensuremath{\mathfrak{i}}}
\providecommand{\Fj}{\ensuremath{\mathfrak{j}}}
\providecommand{\Fk}{\ensuremath{\mathfrak{k}}}
\providecommand{\Fl}{\ensuremath{\mathfrak{l}}}
\providecommand{\Fm}{\ensuremath{\mathfrak{m}}}
\providecommand{\Fn}{\ensuremath{\mathfrak{n}}}
\providecommand{\Fo}{\ensuremath{\mathfrak{o}}}
\providecommand{\Fp}{\ensuremath{\mathfrak{p}}}
\providecommand{\Fq}{\ensuremath{\mathfrak{q}}}
\providecommand{\Fr}{\ensuremath{\mathfrak{r}}}
\providecommand{\Fs}{\ensuremath{\mathfrak{s}}}
\providecommand{\Ft}{\ensuremath{\mathfrak{t}}}
\providecommand{\Fu}{\ensuremath{\mathfrak{u}}}
\providecommand{\Fv}{\ensuremath{\mathfrak{v}}}
\providecommand{\Fw}{\ensuremath{\mathfrak{w}}}
\providecommand{\Fx}{\ensuremath{\mathfrak{x}}}
\providecommand{\Fy}{\ensuremath{\mathfrak{y}}}
\providecommand{\Fz}{\ensuremath{\mathfrak{z}}}

%Provides a list of shortcuts
\usepackage{xspace}
\usepackage{amsmath}
\usepackage{amssymb}
\usepackage{nicefrac}


\providecommand{\catname}[1]{\ensuremath{\mathbf{#1}}\xspace}
\providecommand{\Par}{\catname{Par}}


%Shorthands
\providecommand{\Iso}{Isomorphismus\xspace}
\providecommand{\Ison}{Isomorphismen\xspace}
\providecommand{\Mor}{Morphismus\xspace}
\providecommand{\Morn}{Morphismen\xspace}
\providecommand{\Abb}{Abbildung\xspace}
\providecommand{\Abbn}{Abbildungen\xspace}
%Symbols
\providecommand{\Hom}{\ensuremath{\mathbold{Hom}}}
\providecommand{\Id}{\ensuremath{\mathbb{id}}}

\providecommand{\Ob}{\ensuremath{\mathbold{Ob}}}
%FRAKS
\providecommand{\FA}{\ensuremath{\mathfrak{A}}}
\providecommand{\FB}{\ensuremath{\mathfrak{B}}}
\providecommand{\FC}{\ensuremath{\mathfrak{C}}}
\providecommand{\FD}{\ensuremath{\mathfrak{D}}}
\providecommand{\FE}{\ensuremath{\mathfrak{E}}}
\providecommand{\FF}{\ensuremath{\mathfrak{F}}}
\providecommand{\FG}{\ensuremath{\mathfrak{G}}}
\providecommand{\FH}{\ensuremath{\mathfrak{H}}}
\providecommand{\FI}{\ensuremath{\mathfrak{I}}}
\providecommand{\FJ}{\ensuremath{\mathfrak{J}}}
\providecommand{\FK}{\ensuremath{\mathfrak{K}}}
\providecommand{\FL}{\ensuremath{\mathfrak{L}}}
\providecommand{\FM}{\ensuremath{\mathfrak{M}}}
\providecommand{\FN}{\ensuremath{\mathfrak{N}}}
\providecommand{\FO}{\ensuremath{\mathfrak{O}}}
\providecommand{\FP}{\ensuremath{\mathfrak{P}}}
\providecommand{\FQ}{\ensuremath{\mathfrak{Q}}}
\providecommand{\FR}{\ensuremath{\mathfrak{R}}}
\providecommand{\FS}{\ensuremath{\mathfrak{S}}}
\providecommand{\FT}{\ensuremath{\mathfrak{T}}}
\providecommand{\FU}{\ensuremath{\mathfrak{U}}}
\providecommand{\FV}{\ensuremath{\mathfrak{V}}}
\providecommand{\FW}{\ensuremath{\mathfrak{W}}}
\providecommand{\FX}{\ensuremath{\mathfrak{X}}}
\providecommand{\FY}{\ensuremath{\mathfrak{Y}}}
\providecommand{\FZ}{\ensuremath{\mathfrak{Z}}}
\providecommand{\Fa}{\ensuremath{\mathfrak{a}}}
\providecommand{\Fa}{\ensuremath{\mathfrak{a}}}
\providecommand{\Fb}{\ensuremath{\mathfrak{b}}}
\providecommand{\Fc}{\ensuremath{\mathfrak{c}}}
\providecommand{\Fd}{\ensuremath{\mathfrak{d}}}
\providecommand{\Fe}{\ensuremath{\mathfrak{e}}}
\providecommand{\Ff}{\ensuremath{\mathfrak{f}}}
\providecommand{\Fg}{\ensuremath{\mathfrak{g}}}
\providecommand{\Fh}{\ensuremath{\mathfrak{h}}}
\providecommand{\Fi}{\ensuremath{\mathfrak{i}}}
\providecommand{\Fj}{\ensuremath{\mathfrak{j}}}
\providecommand{\Fk}{\ensuremath{\mathfrak{k}}}
\providecommand{\Fl}{\ensuremath{\mathfrak{l}}}
\providecommand{\Fm}{\ensuremath{\mathfrak{m}}}
\providecommand{\Fn}{\ensuremath{\mathfrak{n}}}
\providecommand{\Fo}{\ensuremath{\mathfrak{o}}}
\providecommand{\Fp}{\ensuremath{\mathfrak{p}}}
\providecommand{\Fq}{\ensuremath{\mathfrak{q}}}
\providecommand{\Fr}{\ensuremath{\mathfrak{r}}}
\providecommand{\Fs}{\ensuremath{\mathfrak{s}}}
\providecommand{\Ft}{\ensuremath{\mathfrak{t}}}
\providecommand{\Fu}{\ensuremath{\mathfrak{u}}}
\providecommand{\Fv}{\ensuremath{\mathfrak{v}}}
\providecommand{\Fw}{\ensuremath{\mathfrak{w}}}
\providecommand{\Fx}{\ensuremath{\mathfrak{x}}}
\providecommand{\Fy}{\ensuremath{\mathfrak{y}}}
\providecommand{\Fz}{\ensuremath{\mathfrak{z}}}

\usepackage{bm}
\title{Ausarbeitung zum Seminar Kategorientheorie}


\begin{document}
\section{Einleitung}
	Die folgende Ausarbeitung ist als Grundlage und Erg\"anzung zu meinem Vortrag zu sehen. 
	Ich werde einige Beispiele hier genauer ausarbeiten, als ich sie in der Praesentation besprechen werde.
	Was ist Kategorientheorie? 
	Ich sehe die Kategorientheorie ein bisschen als eine Art Hilfsmathemathik, so wie andere Wissenschaften die Mathemathik als Hilfswissenschaft ansehen;
	die Mathemathik an sich scheint in vielen Auspr\"agungen die N\"utzlichkeit f\"ur das reale Leben zu fehlen,
	und so mag sich manchem nicht sofort erschliessen, 
	warum die Kategorientheorie so praktisch ist.
	\\
	Im folgenden Teil werde ich einige Definitionen liefern und sie hoffentlich mit den Beispielen von der N\"utzlichkeit der Defininitionen \"uberzeugen koennen. Der Aufbau folgt dem Aufbau der Quelle \nocite{Bra}.
\section{Der Begriff der Kategorie}
	
		Definition(Kategorie)\cite[Definition 2.2.2]{Bra}
		Eine \emph{Kategorie} \CatC besteht aus den folgenden Daten:
		\begin{itemize}
			\item einer Klasse \( \Ob \left( \CatC \right) \), deren Elemente wir \emph{Objekte} nennen,
			\item zu je zwei Objekten 
			\begin{math}
				A,B \in \Ob \left( \CatC  \right) 
			\end{math}
			einer Menge 
			\begin{math}
				\Hom_\CatC \left( A,B \right) 
			\end{math}
			, deren Elemente wir mit 
			\begin{math}
				f : A \to B 
			\end{math}
			notieren und \emph{Morphismen} von $ A $ nach $ B $ nennen,
			\item zu je drei Objekten 
			\begin{math}
		 A,B,C \in \Ob \left( \CatC  \right) 
			\end{math}	
			einer \Abb 
			\begin{displaymath}
				\Hom_\CatC \left( A,B \right) \times \Hom_\CatC \left( B,C \right) \to \Hom_\CatC \left( A,C \right) ,
			\end{displaymath}
			die wir mit 
			\begin{math}
				\left( f,g \right) \mapsto g \circ f
			\end{math}
			notieren und \emph{Komposition von Morphismen} nennen, 
		\item zu jedem Objekt 
			\begin{math}
				A \in \Ob \left( \CatC \right)
			\end{math} 	
			einen ausgezeichneten Morphismus 
			\begin{displaymath}
				\id_A \in \Hom_\CatC ( A,A) ,
			\end{displaymath}
			welchen wir die \emph{identit\"at} nennen.
		\end{itemize}
		Diese Daten m\"ussen den folgenden Regeln gen\"ugen:
		\begin{itemize}
			\item Die Komposition von Morphismen ist \emph{assoziativ}: F\"ur drei Morphismen der Form
			\begin{math}
				f: A \to B , g: B \to C, h:C \to D 
			\end{math}
			in \CatC gilt 
			\begin{displaymath}
				h \circ \left( g \circ f \right) = \left( h \circ g \right) \circ f
			\end{displaymath}
			als Morphismen
			\begin{math}
				A \to D.
			\end{math}
			\item Die identit\"aten sind \emph{beidseitig neutral} \bzgl der Komposition: F\"ur jeden \Mor 
			\begin{math}
				f: A \to B
			\end{math}
			in \CatC gilt
			\begin{displaymath}
				f \circ \id_A = f = \id_B \circ f
			\end{displaymath}
		\end{itemize}
		\
		Was  diese Definition bedeutet werde ich nach kurzem Ausholen an einem Beispiel zeigen.
		Hier erstmal noch einige weiter Definitionen:
		
		%Definition kleine Kategporie
		Eine \emph{kleine Kategorie} ist eine Kategorie, deren Klasse der Objekte eine Menge ist.

		%Definition Start und Ziel
		F\"ur einen Morphismus \( f: A \to B \) in einer Kategorie \CatC nennen wir \( A \) das Start- und \( B \) das Ziel-Objekt.
		
		\label{test}
l		Jetzt folgt das erste Beispiel(\cite[Beispiel 2.2.10]{Bra}):
		
		Zu einem K\"orper \( K \) k\"onnen wir die Kategorie der \( K\)-Vectorr\"aume \( \CatVect_K \) betrachten:
		\begin{itemize}
			\item	\( Ob \CatVect_K = \left\lbrace V | V \text{ist} K-\text{Vektorraum} \right\rbrace  = \left\lbrace K^n | n \in \BN \right\rbrace \)
			\item F\"ur  \( A=K^n , B =K^m , n,m \in \BN \text{aus} Ob\CatVect_K) \) existieren $K$-lineare Abbildungen, diese stellen \( \Hom_{\CatVect_K}\left( A,B \right) \) dar
			\item Die Verkn\"upfung entspricht der Matrixmultiplikation der Abbildungsmatrizen 
			\item \( \id_A = E_n \) f\"ur \( K^n , n \in \BN \)
		 \end{itemize}
		 Die f\"ur die Verkn\"upfung von Morphismen geforderten Eigenschaften folgen sofort aus den Eigenschaften der Matrixmultiplikation.
		 
		 
		 Ein weiteres Beispiel ware eine Pr\"aordnung \( X, \leq \) bestehend aus einer Menge \( X \) und einer bin\"aren Relation \( \leq \), welche reflexiv und transitiv ist.
		 
		 \begin{itemize}
			 \item \(\Ob \) = \BN
			 \item \(  \Hom(A,B) = A \leq B \)
			 \item \( A \leq B \wedge B \leq  C \rightarrow A \leq C \)  (da eine Pr\"aordnung transitiv ist) 
			 \item \( \id_A \): \( A \leq A \) (da eine  Pr\"aordnung reflexiv ist)
		 \end{itemize}
			 Das die zusaetzlichen Gleichungen gelten ist schnell gezeigt:
			 \begin{eqnarray}
			  \text{Assozitivit\"at:} &  ( x \leq y \wedge x \leq z ) & \wedge z \leq a \\
					 \leftrightarrow & (x \leq z ) & \wedge  z \leq a \\
					 \leftrightarrow & x \leq a & \\
					 \leftrightarrow & 	x \leq y  & \wedge  (y \leq a) \\
					 \leftrightarrow &   x \leq y & \wedge  (x \leq z  \wedge  z \leq a )\\
			 \end{eqnarray}
			 und mit den selben Umformungsschritten gilt dies auch f\"ur die Identit\"aten.
		 
	 	Definition (Diskrete und indiskrete Kategorien) \cite[Beispiel 2.2.31]{Bra}
	 		Eine diskrete Kategorie ist eine Kategorie, in der die einzigen Morphismen die Identit\"aten sind.
	 		Wenn in einer Kategorie zwischen allen Objekten genau ein Morphismus existiert, sprechen wir von einer indiskreten Kategorie.
	 		
	 		
		 Beispiel Datenbanken \cite[Beispiel 2.2.33]{Bra}
		 /includestandalone{./files/diagrams/diagram_database}
		 Es wird zusaetzlich gegeben, dass gilt:
		 \begin{itemize}
			 \item \( \text{arbeitet \ in} \circ \text{Abteilungsleiter}= \id_\text{Abteilung}  \)
			 \item \( \text{Abteilungsleiter} \circ \text{arbeitet \ in} \neq \id_\text{Mitarbeiter}  \)
			 \item \( \text{arbeitet \ in }  \circ \text{Manager} = \text{arbeitet \ in } \)
		 \end{itemize}
	\section{Isomorphimen}
	
		Definition (Isomorphismus ) \cite[Definition 2.3.1]{Bra}
		Sei \CatC eine Kategorie. Ein Morphismus \( f: A \to B  \) heisst  \emph{Isomorphismus}, wenn es einen Morphismus \( g: B \to A \) , sodass \( f \circ g  = \id_B \) und \( g \circ f = \id_B \). 
		Dieser Moprhismus \( g \) ist eindeutig und wird mit \( f^{-1} :B \to A \) bezeichnet.
		\(f^{-1} \) wird auch der zu \( f \)  inverse Morphismus genannt.\\
		/includestandalone{./files/diagrams/diagram_isomorph_objects}
		Falls es einen Isomorphismus \( A \to B \) gibt sagen wir \( A \) ist isomorph zu \( B \) und schreiben \( A \equiv B \).
		
		 Definition (Gruppoid) \cite[Beispiel 2.2.34]{Bra}
		 Ein \emph{Gruppoid} ist eine Katergorie, in der jeder \Mor ein \Iso ist.
		  
		  
%##		 Lemma Eigenschaften von Isomorphismen
%##		Isomorphieklassen
%##		Automorphisemngruppe

		 Definition (Homotopieklasse \cite[Aufgabe 2.24]{Bra}
		 Sei X ein topologischer Raum, dann heissen zwei Pfade \( w,w'  \) von \(x \) nach \( y \) homotop, wenn es eine stetige Abbildung \( H : [0,1] \times [0,1 ] \to X \) gibt, sodass \( H(-,0) = w \) , \( H( -,1) =w' \) und \( H(0,t) =x \) und \(H(1,t) = y \forall t \in [0,1] \) gibt.
		 
		 
		 Beispiel(Fundamentalgruppoid)\cite[ Aufgabe 2.24 ]{Bra}
		 Sie X ein topologischer Raum
		\begin{itemize}
			\item \(\Ob \left( \Pi \left( X \right) \right) = x \in X \)
			 \item Seien \( x ,y \in X ) \), dann ist \( \Hom(x,y) \) die Homotopieklasse der Pfade von x nach y
			 \item Die Verkn\"upfung der \Hom \ ist das aneinanderh\"angen von Pfaden.
			 \item \(  \id_x \) ist der konstante Pfad \( w(t)=x \forall t \in [0,1] \)
		\end{itemize}
		Jetzt m\"ussen wir \"uberpr\"ufen, ob die Eigenschaften der Verkn\"upfung gegeben sind.
		Es ist offensichtlich, dass das Vor- oder Nachschalten eines konstanten Weges hier keine Veraenderung bringt, somit gilt f\"ur einen beliebigen Weg 
		\(w:x \to y \) mit \( x,y \in X\):
		\( \id_y \circ w = w = w \circ \id_x \)
		Es bleibt also noch die Assoziativit\"at zuzeigen.
		??
		Da jeder Weg ein Inverses gegeben durch \( w^{-1}(t):=w(1-t): [0,1] \to X \) hat wird diese Kategorie zum Gruppoid.
		Wenn wir jetzt das Fundamentalgruppoid auf einem einzelnen Punkt von \( X \) betrachten, erhalten wir eine Gruppe, diese wird Fundamentalgruppe genannt.
		
	
	\section{Kommutative Diagramme}
		Definition(Kommutatives Diagramm) \cite[Definition 2.4.3]{Bra}
		Sei \( \Gamma =(V,E) \) ein gerichteter Graph. Ein  \emph{Diagramm} $X$ in der Form \( \Gamma \) besteht aus den folgenden Daten:
		\begin{enumerate}
			\item f\"ur jeden Knoten \( v \in V \) ein  Objekt \( X(v) \in \CatC \)
			\item f\"ur jede Kante \( e:v \to w \) in \( E \) ein Morphismus \( X(e):X(v) \to X(w) \)  
		\end{enumerate}
		Das Diagramm $X$ heisst \emph{kommutativ}, wenn f\"ur je zwei Knoten \(v,w \in V \) und je zwei Pfade \( \left(e_n, \dots ,e_1\right) \) und \( \left(f_m, \dots ,f_1\right) \) von $v $ nach $w$ in $\Gamma $ die Gleichung 
		\[
			X(e_n) \circ \dots \circ X(e_1) = X(f_m) \circ \dots \circ X(f_1)
		\]
		von Morphismen \( X(v) \to X(w) \) gilt.
		
		Als Spezialfall hiervon kann man kommutative Dreiecke und Quadrate betrachten:
 		/includestandalone{./files/diagrams/diagram_dreieck_easy}
		/includestandalone{./files/diagrams/diagram_quadrat_easy}		 
		Ein weiteres schoenes Beispiel ist das folgende Diagramm:
		/includestandalone{./files/diagrams/diagram_herz}
		Als nachstes wollen wir uns mit der Diagrammjagd besch\"aftigen \cite[Beispiel 2.4.7]{Bra}.
		Im folgenden Diagramm sind die kleinen Quadrate bereits kommutativ und es ist zu zeigen, dass dann auch schon das grosse Viereck kommutativ ist.
		/includestandalone{./files/diagrams/diagram_doppel_quadrat}
		
		Ein weiteres schoenes Beispiel f\"ur eine Diagrammjagd ist es die Kommutativit\"at beim folgenden W\"urfel \cite[Quellcode anleihe]{tikzcdl} unter der Annahme der Kommutativit\"at der Seitenfl\"achen zu zeigen
		/includestandalone{./files/diagrams/diagram_wuerfel}
		
faktorisierung


\section{Initiale und finale Objekte}
definition
Beispiel 2.5.2 -1

\section{Konstruktion mit Kategorien}
definiton unterkaegiere

defn voll

duale kategorie 
produkte
koprodiukte
ueber A kategorie
kategorie von Wirkungen

\end{document}
