\documentclass{article}
\usepackage[left=3cm,right=3cm]{geometry}
\usepackage{lmodern}
\usepackage[ngerman]{babel}
\usepackage[T1]{fontenc}
\usepackage[ansinew]{inputenc}
\input{./Packages}
\usepackage{tikz}
\usepackage{tikz-cd}
\usetikzlibrary{3d}
\usetikzlibrary{calc}
\usetikzlibrary{arrows}
\usetikzlibrary{babel}

\usepackage{amsmath}
\usepackage{amssymb}
\usepackage{nicefrac}


%Provides a list of shortcuts
\newcommand{\Iso}{Isomorphismus}
\newcommand{\Ison}{Isomorphismen}
\newcommand{\Mor}{Morphismus}
\newcommand{\Morn}{Morphismen}

%Provides a list of shortcuts
\providecommand{\Iso}{Isomorphismus\xspace}
\providecommand{\Ison}{Isomorphismen\xspace}
\providecommand{\Mor}{Morphismus\xspace}
\providecommand{\Morn}{Morphismen\xspace}

%Standart Bibliography entries
\usepackage[fixlanguage]{babelbib}
\selectbiblanguage{german}

@book{Bra,
author = {Martin Brandenburg},
title = {Einf\"uhrung in die Kategorientheorie, mit ausf\"uhrlichen Erkl\"arungen und Beispielen},
date = {2015},
publisher = {Springer},
isbn = {978-3-662-47067-1},
doi = {10.1007/978-3-662-47068-8},
}

\title{Ausarbeitung zum Seminar Kategorientheorie}
\bibliography{Ausarbeitung}


\begin{document}
\section{Einleitung}
	Die folgende Ausarbeitung ist als Grundlage und Erg\"anzung zu meinem Vortrag zu sehen. 
	Ich werde einige Beispiele hier genauer ausarbeiten, als ich sie in der Praesentation besprechen werde.
	Was ist Kategorientheorie? 
	Ich sehe die Kategorientheorie ein bisschen als eine Art Hilfsmathemathik, so wie andere Wissenschaften die Mathemathik als Hilfswissenschaft ansehen;
	die Mathemathik an sich scheint in vielen Auspr\"agungen die N\"utzlichkeit f\"ur das reale Leben zu fehlen,
	und so mag sich manchem nicht sofort erschliessen, 
	warum die Kategorientheorie so praktisch ist.
	\\
	Im folgenden Teil werde ich einige Definitionen liefern und sie hoffentlich mit den Beispielen von der N\"utzlichkeit der Defininitionen \"uberzeugen koennen. Der Aufbau folgt dem Aufbau der Quelle \nocite{Bra}.
\section{Der Begriff der Kategorie}
	
		%Definition 2.2.2
		Eine \emph{Kategorie} \CatC besteht aus den folgenden Daten:
		\begin{itemize}
			\item einer Klasse $ \Ob \left( \CatC \right)$, deren Elemente wir \emph{Objekte} nennen,
			\item zu je zwei Objekten 
			\begin{math}
				A,B \in \Ob \left( \CatC  \right) 
			\end{math}
			einer Menge 
			\begin{math}
				\Hom_\CatC \left( A,B \right) 
			\end{math}
			, deren Elemente wir mit 
			\begin{math}
				f : A \to B 
			\end{math}
			notieren und \emph{Morphismen} von $ A $ nach $ B $ nennen,
			\item zu je drei Objekten 
			\begin{math}
		 A,B,C \in \Ob \left( \CatC  \right) 
			\end{math}	
			einer \Abb 
			\begin{displaymath}
				\Hom_\CatC \left( A,B \right) \times \Hom_\CatC \left( B,C \right) \to \Hom_\CatC \left( A,C \right) ,
			\end{displaymath}
			die wir mit 
			\begin{math}
				\left( f,g \right) \mapsto g \circ f
			\end{math}
			notieren und \emph{Komposition von Morphismen} nennen, 
	\right) \right) 		\item zu jedem Objekt 
			\begin{math}
				A \in \Ob \left( \CatC \right)
			\end{math} 	
			einen ausgezeichneten Morphismus 
			\begin{displaymath}
				\Id_A \in \Hom_\CatC ( A,A) ,
			\end{displaymath}
			welchen wir die \emph{Identit\"at} nennen.
		\end{itemize}
		Diese Daten m\"ussen den folgenden Regeln gen\"ugen:
		\begin{itemize}
			\item Die Komposition von Morphismen ist \emph{assoziativ}: F\"ur drei Morphismen der Form
			\begin{math}
				f: A \to B , g: B \to C, h:C \to D 
			\end{math}
			in \CatC gilt 
			\begin{displaymath}
				h \circ \left( g \circ f \right) = \left( h \circ g \right) \circ f
			\end{displaymath}
			als Morphismen
			\begin{math}
				A \to D.
			\end{math}
			\item Die Identit\"aten sind \emph{beidseitig neutral} \bzgl der Komposition: F\"ur jeden \Mor 
			\begin{math}
				f: A \to B
			\end{math}
			in \CatC gilt
			\begin{displaymath}
				f \circ \Id_A = f = \Id_B \circ f
			\end{displaymath}
		\end{itemize}
		\\
		Was  diese Definition bedeutet werde ich nach kurzem Ausholen an einem Beispiel zeigen.
		Hier erstmal noch einige weiter Definitionen:
		
		%Definition kleine Kategporie
		Eine \emph{kleine Kategorie} ist eine Kategorie, deren Klasse der Objekte eine Menge ist.

		%Definition Start und Ziel
		F\"ur einen Morphismus \( f: A \to B \) in einer Kategorie \CatC nennen wir \( A \) das Start- und \( B \) das Ziel-Objekt.
		
		\label{test}
		Jetzt folgt das erste Beispiel(\cite[Beispiel 2.2.10]{Bra}):
		
		Zu einem K\"orper \( K \) k\"onnen wir die Kategorie der \( K\)-Vectorr\"aume \( \CatVect_K \) betrachten:
		\begin{itemize}
			\item	\( Ob \CatVect_K = \left\lbrace V | V \text{ist} K-\text{Vektorraum} \right\rbrace  = \left\lbrace K^n | n \in \BN \right\rbrace \)
			\item F\"ur  \( A=K^n , B =K^m , n,m \in \BN \text{aus} Ob\CatVect_K) \) existieren $K$-lineare Abbildungen, diese stellen \( \Hom_{\CatVect_K}\left( A,B \right) \) dar
			\item Die Verkn\"upfung entspricht der Matrixmultiplikation der Abbildungsmatrizen (hierraus folgt )
\bibliographystyle{plain}
\bibliography{sample1,sample2,...,samplen} 
% Note the lack of whitespace between the commas and the next bib file.
\end{document}
