% !TeX spellcheck = de_DE
% !TeX encoding = utf8


\documentclass{article}
\usepackage[left=3cm,right=3cm]{geometry}
\usepackage{lmodern}
\usepackage[ngerman]{babel}
\usepackage[T1]{fontenc}
\usepackage[ansinew]{inputenc}
\include{./packages}
\usetikzlibrary{3d,calc}
\usetikzlibrary{arrows}
\usepackage{amsmath}
\usepackage{amssymb}
\usepackage{bm}
\usepackage{nicefrac}
\usepackage{mathepakete}
\usepackage{/usr/local/share/texmf/tex/latex/shortcuts}
\usepackage{shortcuts}
\renewcommand{\ez}{\mathbf{1}}
\DeclareBoldMathCommand{\T}{T}
\DeclareBoldMathCommand{\p}{p}

\title{\"Ubungen zum Seminar \boldfont{Kategorientheorie} }

\author{Jens Heinrich}
\date{16. Juni 2015}

 
\begin{document}
 

\section*{Aufgabe 2.1}
	In einem Dreieck 

	%Insert Diagram here
	% Tikz file for commutative diagramm
% this provides a universal header for all tikz 
% standalone files
% it may become crouded so this is not necessary 
% the final document layout
\documentclass{standalone}
\input{./Packages}
\usepackage{tikz}
\usepackage{tikz-cd}
\usetikzlibrary{3d}
\usetikzlibrary{calc}
\usetikzlibrary{arrows}
\usetikzlibrary{babel}

\usepackage{amsmath}
\usepackage{amssymb}
\usepackage{nicefrac}


%Provides a list of shortcuts
\newcommand{\Iso}{Isomorphismus}
\newcommand{\Ison}{Isomorphismen}
\newcommand{\Mor}{Morphismus}
\newcommand{\Morn}{Morphismen}

%Provides a list of shortcuts
\providecommand{\Iso}{Isomorphismus\xspace}
\providecommand{\Ison}{Isomorphismen\xspace}
\providecommand{\Mor}{Morphismus\xspace}
\providecommand{\Morn}{Morphismen\xspace}

\usepackage{xspace}
\input{./Packages_math}
\providecommand{\catname}[1]{\ensuremath{\mathbf{#1}}\xspace}
\providecommand{\Par}{\catname{Par}}



\begin{document}
	\begin{tikzcd}[row sep=30pt, column sep=25pt]
		A \ar{rr}{f} \ar{dr}[swap]{g} 
		&
		& 
		B \ar{dl}{h}\\
		&
		C
		&
	\end{tikzcd}
\end{document}

	in einer Kategorie seinen zwei der drei Morphismen Isomorphisen.
	Zeige, dass dann der dritte Morphismus ebenfalls ein Isomorphismus ist.

		Wir beginnen mit einer Fallunterscheidung, danach welcher der drei Morphismen noch kein Isomorphismus ist.

	\include{./files/diagrams/example_diagram.tikz}

\end{document}
