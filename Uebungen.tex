% !TeX spellcheck = de_DE
% !TeX encoding = utf8


\documentclass{article}
\usepackage[left=3cm,right=3cm]{geometry}
\usepackage{lmodern}
\usepackage[ngerman]{babel}
\usepackage[T1]{fontenc}
\usepackage[ansinew]{inputenc}
\usepackage{tikz}
\usetikzlibrary{matrix}

\usetikzlibrary{3d,calc}
\usetikzlibrary{arrows}
\usepackage{amsmath}
\usepackage{amssymb}
\usepackage{bm}
\usepackage{nicefrac}
\usepackage{mathepakete}
\usepackage{/usr/local/share/texmf/tex/latex/shortcuts}
\usepackage{shortcuts}
\renewcommand{\ez}{\mathbf{1}}
\DeclareBoldMathCommand{\T}{T}
\DeclareBoldMathCommand{\p}{p}

\title{\"Ubungen zum Seminar \boldfont{Kategorientheorie} }

\author{Jens Heinrich}
\date{16. Juni 2015}

 
\begin{document}
 

\section*{Aufgabe 2.1}
	In einem Dreieck 

	%Insert Diagram here
	\includeonly{./files/diagrams/diagramm_dreieck_easy.tikz}
	in einer Kategorie seinen zwei der drei Morphismen Isomorphisen.
	Zeige, dass dann der dritte Morphismus ebenfalls ein Isomorphismus ist.

		Wir beginnen mit einer Fallunterscheidung, danach welcher der drei Morphismen noch kein Isomorphismus ist.

	% this provides a universal header for all tikz 
% standalone files
% it may become crouded so this is not necessary 
% the final document layout
\documentclass{standalone}
\usepackage{tikz}
\usetikzlibrary{3d,calc,arrows}

\begin{document} 
	\begin{tikzpicture}
  	\matrix (m) [matrix of math nodes,row sep=3em,column sep=4em,minimum width=2em]
  	{
     		F_t(x) & F(x) \\
     		A_t & A \\};
  	\path[-stealth]
    		(m-1-1) edge node [left] {$\mathcal{B}_X$} (m-2-1)
            	edge [double] node [below] {$\mathcal{B}_t$} (m-1-2)
    		(m-2-1.east|-m-2-2) edge node [below] {$\mathcal{B}_T$}
            	node [above] {$\exists$} (m-2-2)
    		(m-1-2) edge node [right] {$\mathcal{B}_T$} (m-2-2)
            	edge [dashed,-] (m-2-1);
	\end{tikzpicture}
\end{document}


\end{document}
