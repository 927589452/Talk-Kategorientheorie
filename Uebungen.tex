% !TeX spellcheck = de_DE
% !TeX encoding = utf8


\documentclass{article}
\usepackage[left=3cm,right=3cm]{geometry}
\usepackage{lmodern}
\usepackage[ngerman]{babel}
\usepackage[T1]{fontenc}
\usepackage[ansinew]{inputenc}
\input{./Packages}
\usepackage{tikz}
\usepackage{tikz-cd}
\usetikzlibrary{3d}
\usetikzlibrary{calc}
\usetikzlibrary{arrows}
\usetikzlibrary{babel}

\usepackage{amsmath}
\usepackage{amssymb}
\usepackage{nicefrac}


\usepackage{bm}
\usepackage{mathepakete}
\usepackage{/usr/local/share/texmf/tex/latex/shortcuts}
\usepackage{shortcuts}

\title{\"Ubungen zum Seminar \boldfont{Kategorientheorie} }

\author{Jens Heinrich}
\date{16. Juni 2015}

 
\begin{document}
 

\section*{Aufgabe 2.1}
	In einem Dreieck 

	%Insert Diagram here
	\includestandalone{./files/diagrams/diagramm_dreieck_easy}
	in einer Kategorie seinen zwei der drei Morphismen Isomorphisen.
	Zeige, dass dann der dritte Morphismus ebenfalls ein Isomorphismus ist.
	\underline{Beweis:} \\
		Wir beginnen mit einer Fallunterscheidung, danach welcher der drei Morphismen noch kein Isomorphismus ist.
		

\end{document}
