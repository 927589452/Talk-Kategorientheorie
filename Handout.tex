% !TeX spellcheck = de_DE
% !TeX encoding = utf8


\documentclass{article}
\usepackage[left=3cm,right=3cm]{geometry}
\usepackage{lmodern}
\usepackage[ngerman]{babel}
\usepackage[T1]{fontenc}
\usepackage[ansinew]{inputenc}
\input{./Packages}
\usepackage{tikz}
\usepackage{tikz-cd}
\usetikzlibrary{3d}
\usetikzlibrary{calc}
\usetikzlibrary{arrows}
\usetikzlibrary{babel}

\usepackage{amsmath}
\usepackage{amssymb}
\usepackage{nicefrac}


%Provides a list of shortcuts
\newcommand{\Iso}{Isomorphismus}
\newcommand{\Ison}{Isomorphismen}
\newcommand{\Mor}{Morphismus}
\newcommand{\Morn}{Morphismen}

%Provides a list of shortcuts
\providecommand{\Iso}{Isomorphismus\xspace}
\providecommand{\Ison}{Isomorphismen\xspace}
\providecommand{\Mor}{Morphismus\xspace}
\providecommand{\Morn}{Morphismen\xspace}

\usepackage{bm}



\title{Handout zum Seminar Kategorientheorie}
\author{Jens Heinrich}
\date{14.04.2016}

\begin{document}
\maketitle
\begin{defi}[Kategorie]
\section{Der Begriff der Kategorie}
		\cite[Definition 2.2.2]{Bra}
		Eine \emph{Kategorie} \CatC besteht aus den folgenden Daten:
		\begin{itemize}
			\item einer Klasse \( \Ob \left( \CatC \right) \), deren Elemente wir \emph{Objekte} nennen,
			\item zu je zwei Objekten 
			\begin{math}
				A,B \in \Ob \left( \CatC  \right) 
			\end{math}
			einer Menge 
			\begin{math}
				\Hom_\CatC \left( A,B \right) 
			\end{math}
			, deren Elemente wir mit 
			\begin{math}
				f : A \to B 
			\end{math}
			notieren und \emph{Morphismen} von $ A $ nach $ B $ nennen,
			\item zu je drei Objekten 
			\begin{math}
		 A,B,C \in \Ob \left( \CatC  \right) 
			\end{math}	
			einer \Abb 
			\begin{displaymath}
				\Hom_\CatC \left( A,B \right) \times \Hom_\CatC \left( B,C \right) \to \Hom_\CatC \left( A,C \right) ,
			\end{displaymath}
			die wir mit 
			\begin{math}
				\left( f,g \right) \mapsto g \circ f
			\end{math}
			notieren und \emph{Komposition von Morphismen} nennen, 
		\item zu jedem Objekt 
			\begin{math}
				A \in \Ob \left( \CatC \right)
			\end{math} 	
			einen ausgezeichneten Morphismus 
			\begin{displaymath}
				\id_A \in \Hom_\CatC ( A,A) ,
			\end{displaymath}
			welchen wir die \emph{Identit\"at} nennen.
		\end{itemize}
			Diese Daten m\"ussen den folgenden Regeln gen\"ugen:
			\begin{itemize}
				\item Die Komposition von Morphismen ist \emph{assoziativ}: F\"ur drei Morphismen der Form
				\begin{math}
					f: A \to B , g: B \to C, h:C \to D 
				\end{math}
				in \CatC gilt 
				\begin{displaymath}
					h \circ \left( g \circ f \right) = \left( h \circ g \right) \circ f
				\end{displaymath}
				als Morphismen
				\begin{math}
					A \to D.
				\end{math}
				\item Die Identit\"aten sind \emph{beidseitig neutral} \bzgl der Komposition: F\"ur jeden \Mor 
				\begin{math}
					f: A \to B
				\end{math}
				in \CatC gilt
				\begin{displaymath}
					f \circ \id_A = f = \id_B \circ f
				\end{displaymath}
			\end{itemize}
		\end{defi}
		\newpage
		\section{Isomorphismen}
		
		\begin{lem}[Eigenschaften von Isomorphismen]
		   \cite[Lemma 2.3.9]{Bra}
		  Es sei \CatC eine Kategorie.
		  \begin{enumerate}
			  \item \( \forall A \in \CatC \) ist \( \id_A :A \to A  \) ein Isomorphismus mit \( \id_A^{-1}=\id_A \)
			  \item F\"ur jeden Isomorphismus \( f: A \to B \) in \CatC gilt: Der inverse Morphismus \(f^{-1}: B \to A \) ist ebenfalls ein Isomorphismus mit \( \left( f^{-1} \right)^{-1} \).
			  \item Ist \(g: B \to C \) ein weiterer Isomorphismus, so ist \( g \circ f :A \to C \) ebenfalls ein Isomorphismus mit \( \left( g \circ f \right)^{-1} =f^{-1} \circ g^{-1} \)  
		  \end{enumerate}
		  	\end{lem}
		  	
		\section{Kommutative Diagramme}
		
		 \includestandalone{./files/diagrams/diagram_doppel_quadrat}
		\includestandalone{./files/diagrams/diagram_wuerfel} \nocite{tikzcd}
		
		\section{Initiale und finale Objekte}
		\includestandalone{./files/diagrams/diagram_teiler_6} 
		\includestandalone{./files/diagrams/diagram_teiler_dual}
		
		%%%%
		\bibliography{sources}
		\bibliographystyle{alpha}		
\end{document}	
