
In einem Dreieck 
	\\
	%Insert Diagram here
	\includestandalone{./files/diagrams/diagramm_dreieck_easy}
	\\
	in einer Kategorie seinen zwei der drei Morphismen \Ison.
	Zeige, dass dann der dritte Morphismus ebenfalls ein \Iso \ ist.
	\\
	\underline{Beweis:} \\
		Wir beginnen mit einer Fallunterscheidung, danach welcher der drei \Morn \ noch kein \Iso \ ist.
		\\
		\underline{h kein \Iso :}
		\\
		Sei 
		\begin{math}
			\Phi=f \circ g^{-1}
		\end{math}		
		, dann gilt :
		\begin{itemize}
			\item 
				\begin{math}
					\Phi : C \to B
				\end{math}
			\item
				\begin{math}
					\Phi \circ h \overset{ kommutiert}{=}
					\Phi \circ \left( 
						g \circ f^{-1} 
						\right) =
					\left( f \circ g^{-1} \right) 
					\circ 
					\left( g \circ f^{-1} \right)= 
					id_{B}
				\end{math}
			\item
				\begin{math}
					h \circ \Phi \overset{ kommutiert}{=}
					\left( 
						g \circ f^{-1}
						\right) \circ \Phi =
					\left( g \circ f^{-1} \right)
					\circ
					\left( f \circ g^{-1} \right)=
					id_{C}
				\end{math}
		\end{itemize}
		\\
		Somit ist 
		\begin{math}
			\Phi
		\end{math}
		\ das Inverse zu 
		\begin{math}
			h
		\end{math}
		. 
		Also ist 
		\begin{math}
			h
		\end{math}
		ein \Iso. 
		\\
		\underline {g kein \Iso :}
		\\
		Sei 
		\begin{math}
			\Phi=f^{-1} \circ h^{-1}
		\end{math}		
		, dann gilt :
		\begin{itemize}
			\item 
				\begin{math}
					\Phi : C \to A
				\end{math}
			\item
				\begin{math}
					\Phi \circ g \overset{ kommutiert}{=}
					\Phi \circ \left( 
						h \circ f  
						\right) =
					\left( f^{-1} \circ h^{-1} \right) 
					\circ 
					\left( h \circ f  \right)= 
					id_{A}
				\end{math}
			\item
				\begin{math}
					g \circ \Phi \overset{ kommutiert}{=}
					\left( 
						h \circ f
 						\right) \circ \Phi =
					\left( h \circ f  \right)
					\circ
					\left( f^{-1} \circ h^{-1} \right)=
					id_{C}
				\end{math}
		\end{itemize}
		\\
		Somit ist 
		\begin{math}
			\Phi
		\end{math}
		\ das Inverse zu 
		\begin{math}
			g
		\end{math}
		. 
		Also ist 
		\begin{math}

			g
		\end{math}
		ein \Iso. 
		\\
		\underline {f kein \Iso :}
		\\
		Sei 
		\begin{math}
			\Phi=g^{-1} \circ h
		\end{math}		
		, dann gilt :
		\begin{itemize}
			\item 
				\begin{math}
					\Phi : B \to A
				\end{math}
			\item
				\begin{math}
					\Phi \circ f \overset{ kommutiert}{=}
					\Phi \circ \left( 
						g^{-1} \circ h  
						\right) =
					\left(  h^{-1} \circ g \right) 
					\circ 
					\left( g^{-1} \circ h )= 
					id_{A}
				\end{math}
			\item
				\begin{math}
					f \circ \Phi \overset{ kommutiert}{=}
					\left( 
						h^{-1} \circ g
 						\right) \circ \Phi =
					\left(h^{-1} \circ g \right)
					\circ
					\left( g^{-1} \circ h  \right)=
					id_{B}
				\end{math}
		\end{itemize}
		\\
		Somit ist 
		\begin{math}
			\Phi
		\end{math}
		\ das Inverse zu 
		\begin{math}
			f
		\end{math}
		. 
		Also ist 
		\begin{math}
			f
		\end{math}
		ein \Iso. 
		\\		
